% Anpassung des Seitenlayouts --------------------------------------------------
%   siehe Seitenstil.tex
% ------------------------------------------------------------------------------
\usepackage[
    automark, % Kapitelangaben in Kopfzeile automatisch erstellen
    headsepline, % Trennlinie unter Kopfzeile
    ilines % Trennlinie linksb�ndig ausrichten
]{scrpage2}

% Anpassung an Landessprache ---------------------------------------------------
\usepackage[ngerman]{babel}

% Umlaute ----------------------------------------------------------------------
%   Umlaute/Sonderzeichen wie ���� direkt im Quelltext verwenden (CodePage).
%   Erlaubt automatische Trennung von Worten mit Umlauten.
% ------------------------------------------------------------------------------
\usepackage[latin1]{inputenc}
\usepackage[T1]{fontenc}
\usepackage{textcomp} % Euro-Zeichen etc.

% Schrift ----------------------------------------------------------------------
\usepackage{lmodern} % bessere Fonts
\usepackage{relsize} % Schriftgr��e relativ festlegen

% Grafiken ---------------------------------------------------------------------
% Einbinden von JPG-Grafiken erm�glichen
\usepackage[dvips,final]{graphicx}
% hier liegen die Bilder des Dokuments
\graphicspath{{Bilder/}}

% Befehle aus AMSTeX f�r mathematische Symbole z.B. \boldsymbol \mathbb --------
\usepackage{amsmath,amsfonts}

% f�r Index-Ausgabe mit \printindex --------------------------------------------
\usepackage{makeidx}

% Einfache Definition der Zeilenabst�nde und Seitenr�nder etc. -----------------
\usepackage{setspace}
\usepackage{geometry}

%option acronym um abk�rzungen leichter zu erstellen
\usepackage[acronym=true, nonumberlist, toc]{glossaries}

% zum Umflie�en von Bildern ----------------------------------------------------
\usepackage{floatflt}

\usepackage[bottom, hang]{footmisc}
\renewcommand{\footnotemargin}{0.3cm}

% zum Einbinden von Programmcode -----------------------------------------------
\usepackage{listings}
\usepackage{xcolor} 
\definecolor{hellgelb}{rgb}{1,1,0.9}
\definecolor{colKeys}{rgb}{0,0,1}
\definecolor{colIdentifier}{rgb}{0,0,0}
\definecolor{colComments}{rgb}{1,0,0}
\definecolor{colString}{rgb}{0,0.5,0}
%\lstset{
%    float=htbp,
%    numbers=none,
%    basicstyle=\ttfamily\color{black}\small\smaller,
%    identifierstyle=\color{colIdentifier},
%    keywordstyle=\color{colKeys},
%    stringstyle=\color{colString},
%    commentstyle=\color{colComments},
%    columns=flexible,
%    tabsize=2,
%    numberstyle=false,
%    extendedchars=true,
%    showspaces=false,
%    showstringspaces=false,
%    breaklines=true,
%    breakautoindent=true,
%    captionpos=b,
%    xleftmargin=17pt
%    }
\lstset{
    float=htbp,
    basicstyle=\ttfamily\smaller,
    columns=flexible,
    tabsize=2,
    extendedchars=true,
    showspaces=false,
    showstringspaces=false,
    numbers=left,
    numberstyle=\tiny,
    breaklines=true,
    breakautoindent=true,
    keywordstyle=\ttfamily\smaller,
    captionpos=b,
    xleftmargin=10pt,
    xrightmargin=10pt,
}

% wichtig f�r korrekte Zitierweise ---------------------------------------------
%\usepackage[square]{natbib}

% PDF-Optionen -----------------------------------------------------------------
\usepackage[
    bookmarks,
    bookmarksopen=true,
    colorlinks=true,
% diese Farbdefinitionen zeichnen Links im PDF farblich aus
%    linkcolor=red, % einfache interne Verkn�pfungen
%    anchorcolor=black,% Ankertext
%    citecolor=blue, % Verweise auf Literaturverzeichniseintr�ge im Text
%    filecolor=magenta, % Verkn�pfungen, die lokale Dateien �ffnen
%    menucolor=red, % Acrobat-Men�punkte
%    urlcolor=cyan, 
% diese Farbdefinitionen sollten f�r den Druck verwendet werden (alles schwarz)
    linkcolor=black, % einfache interne Verkn�pfungen
    anchorcolor=black, % Ankertext
    citecolor=black, % Verweise auf Literaturverzeichniseintr�ge im Text
    filecolor=black, % Verkn�pfungen, die lokale Dateien �ffnen
    menucolor=black, % Acrobat-Men�punkte
    urlcolor=black, 
    plainpages=false, % zur korrekten Erstellung der Bookmarks
    pdfpagelabels, % zur korrekten Erstellung der Bookmarks
    hypertexnames=false, % zur korrekten Erstellung der Bookmarks
    linktocpage % Seitenzahlen anstatt Text im Inhaltsverzeichnis verlinken
]{hyperref}
% Befehle, die Umlaute ausgeben, f�hren zu Fehlern, wenn sie hyperref als Optionen �bergeben werden
\hypersetup{
    pdftitle={\titel},
    pdfauthor={\autor},
    pdfcreator={\autor},
    pdfsubject={\titel},
    pdfkeywords={\titel},
}

% fortlaufendes Durchnummerieren der Fu�noten ----------------------------------
\usepackage{chngcntr}

% f�r lange Tabellen -----------------------------------------------------------
\usepackage{longtable}
\usepackage{array}
\usepackage{ragged2e}
\usepackage{lscape}

% Spaltendefinition rechtsb�ndig mit definierter Breite ------------------------
\newcolumntype{w}[1]{>{\raggedleft\hspace{0pt}}p{#1}}

% Formatierung von Listen �ndern -----------------------------------------------
\usepackage{paralist}

% bei der Definition eigener Befehle ben�tigt
\usepackage{ifthen}

% definiert u.a. die Befehle \todo und \listoftodos
\usepackage{todonotes}

% sorgt daf�r, dass Leerzeichen hinter parameterlosen Makros nicht als Makroendezeichen interpretiert werden
\usepackage{xspace}

% beseitigt Warnungen zu floats mit KOMA-Script
\usepackage{scrhack} 

% Captions linksb�ndig
%\usepackage[justification=raggedright,singlelinecheck=false]{caption}
